\documentclass{EESD}
% To change the slides size go to EESD.cls file and edit the preamble as explained.

% ---- Add your Meta-data to the PDF (Copyrights Kinda!) ----
\hypersetup{
  pdfinfo={
    Title={Presentation: 3D finite element modeling of historical masonry walls},
    Author={Mahmoud S. Shaqfa, Katrin Beyer},
    Subject={EPFL - ENAC - EESD Lab},
    Keywords={Stone masonry, Detailed micro-mechanical, 3D micro-structure}
  }
}

% Important packages to be called
\usepackage{subcaption} % for adding sub-figures
\usepackage{graphicx}
\usepackage[absolute,overlay]{textpos} % To place the figures by coordinates (x,y) - Beamer doesn't support floats XD
\usepackage{multicol} % To adjust items and stuff automatically in a number of a pre-specified columns
\graphicspath{{Figures/}}
\usepackage[utf8]{inputenc}
\usepackage{amsmath}
\usepackage{amsfonts}
\usepackage{amssymb}
\usepackage{lipsum} % Just a dummy text generator
\usepackage{hyperref}
% fonts packages
\usepackage{ragged2e} % Justified typesetting
\usepackage{tgbonum} % fonts

% For References Only
\usepackage[style=authortitle,backend=bibtex]{biblatex}
\addbibresource{References.bib} % Call the references database
\AtBeginBibliography{\tiny} % Specify font size (Size matters)
\renewcommand{\footnotesize}{\tiny}



\setbeamertemplate{footline}
{
  \vspace{-1.5ex}
  \leavevmode
  \hbox{
  \hspace{-2ex}

  \begin{beamercolorbox}[wd=.333333\paperwidth, ht=2.25ex,dp=1.25ex,center]{footer color3}%
    \usebeamerfont{author in head/foot}\insertshortauthor
  \end{beamercolorbox}%
  \begin{beamercolorbox}[wd=.333333\paperwidth, ht=2.25ex, dp=1.25ex, center]{footer color2}%
    \usebeamerfont{header color}\insertshorttitle
  \end{beamercolorbox}%
  \begin{beamercolorbox}[wd=.343333\paperwidth, ht=2.25ex, dp=1.25ex, right]{footer color1}%
    \usebeamerfont{footer color3}\insertshortdate{}\hspace*{6em}
    \insertframenumber{} / \inserttotalframenumber\hspace*{2ex} 
  \end{beamercolorbox}
  }
  \vskip0pt%
}


\author{EESD Official Template}
\title[Your tiny title here]{My long, detailed, and flashy title}

\institute[ENAC]{{\'Ecole Polytechnique F\'ed\'erale de Lausanne (EPFL)}{\newline\newline School of Architecture, Civil and Environmental Engineering (ENAC)}}
\subject{Candidacy Exam}
\date{May 2019}

\begin{document}
{
\usebackgroundtemplate{\includegraphics[width=1.\paperwidth, height=1.\paperheight]{cover169.pdf}} % To add a background for this slide XD -  change it
% -----------------------Cover page
\begin{frame}
\titlepage{~}
{\newline Supervisor: Prof. Katrin Beyer}
\end{frame}
}

\setbeamertemplate{logo}{} % To override the logo from the other slides

% -----------------------Table of contents TOC Three Styles

% Explicitly split the TOC if it's too long
\begin{frame}[allowframebreaks]{Outlines}
\tableofcontents[sections={1-3}] % Explicitly split TOC
\framebreak
\tableofcontents[sections={4-7}] % Explicitly split TOC
\end{frame}
\normalsize

% Just a normal TOC 
\begin{frame}[allowframebreaks]{Outlines}
\tableofcontents
\end{frame}
\normalsize

% Use smart division for the TOC
\begin{frame}[allowframebreaks]{Outlines}
\begin{multicols}{2}
\tableofcontents
\end{multicols}
\end{frame}
\normalsize

% Define and customize the headline style of slides
\setbeamertemplate{headline}{
\leavevmode
  \hbox{
    % if you use aspect ratio 16:9 wd = 1.009 to fill the sides of the bar
    % if you use aspect ratio 4:6 wd = 1.012 to fill the sides of the bar
    \begin{beamercolorbox}[wd=1.009\paperwidth, ht=2.5ex, dp=1.125ex]{header color}
    \ifx\insertsubsection\empty % no subsection
        {~\insertsection}
    \else % subsection exists
    \ifx\insertsubsubsection\empty % subsection but no subsubsection
        {~\insertsection \textcolor{white}{$~~\bullet$} ~\S~\insertsubsection}
    \else % subsection and subsubsection exist
        {~\insertsection \textcolor{white}{$~~\bullet$} ~\S~\insertsubsection \textcolor{white}{$~~\bullet$} ~\S~\insertsubsubsection}
    \fi
    \fi
\end{beamercolorbox}
  }
}




% -----------------------Introduction


\section{Introduction}


\breakingframe{
\begin{textblock*}{3cm}[0.5,0.5](0.5\textwidth,  0.5\textheight)
\Huge\textbf{\textcolor{black}{Introduction}}
\end{textblock*}
}

\subsection{Copyright}
\begin{frame}[t]{Copyright}
\begin{textblock*}{13cm}(1.0cm,  1.7cm)
    {\fontfamily{qcr}\selectfont
    This file is a customized "beamer" template made for the EESD laboratory at EPFL (see \href{https://www.epfl.ch/labs/eesd/}{https://www.epfl.ch/labs/eesd/}). The author of this file is \textbf{Mahmoud S. Shaqfa}. This file is free: you can redistribute it and/or modify it under the terms of the GNU General Public License as published by    the Free Software Foundation, either version 3 of the License, or (at your option) any later version. This file is distributed in the hope that it will be useful, but WITHOUT ANY WARRANTY; without even the implied warranty of MERCHANTABILITY or FITNESS FOR A PARTICULAR PURPOSE.  See the GNU General Public License for more details. To receive a copy of GNU License refer to: \href{https://www.gnu.org/licenses/}{https://www.gnu.org/licenses/}
    }
\end{textblock*}
\end{frame}

\begin{frame}
\frametitle{Blocky blocky}

\begin{block}{Just a Block}
\lipsum[1]
\end{block}

\end{frame}

\begin{frame}
\frametitle{Blocky blocky}

\begin{exampleblock}{Example Block}
\lipsum[1]
\end{exampleblock}

\end{frame}


\begin{frame}
\frametitle{Blocky blocky}

\begin{alertblock}{Alert Block}
\lipsum[1]
\end{alertblock}

\end{frame}

{
\usebackgroundtemplate{\includegraphics[width=1.\paperwidth]{Coordinates.pdf}}
\setbeamertemplate{headline}{}
\setbeamertemplate{footline}{}

\begin{frame}[t]{}

\end{frame}
}


\subsection{Preamble}
\begin{frame}[t]{Preamble}\vspace{4pt}
\begin{itemize}
    \item Historical structures are part of the cultural heritage\vspace{10pt}\pause
    \item Stone masonry is one of the oldest construction materials\vspace{10pt}\pause
    \item Historical stone masonry structures are designed to handle gravity loads\vspace{10pt}\pause
    \item Stone masonry structure are vulnerable under seismic actions:\vspace{5pt}
    \begin{itemize}
        \item Low tensile strength
        \item Poor interlocking
        \item Large masses
        \item Built with rules-of-thumb
    \end{itemize}
\end{itemize}
\end{frame}


\usebackgroundtemplate{\includegraphics[width=1.\paperwidth]{break.pdf}}
\begin{frame}[t]{}
\begin{textblock*}{10cm}(4.7cm,2.8cm)
\Huge\textbf{\textcolor{black}{State-of-the-art}}
\end{textblock*}
\begin{textblock*}{10cm}(1.5cm,4.8cm)
\small\textbf{\textcolor{black}{{\cite{REF:5} (2018)}
}}
\end{textblock*}
\end{frame}
\usebackgroundtemplate{}

\begin{frame}[t]{The research questions}\vspace{10pt}
    \begin{textblock*}{13cm}(1.3cm,2.8cm)
        \begin{itemize}
            \item[\textbf{Q1}] What are the expected gains of using 3D micro-mechanical modeling of stone masonry walls?\vspace{10pt}\pause
            \item[\textbf{Q2}] What tools/methods will be used to relax the 3D FE models' complexity?\vspace{20pt}\pause
            \item[\textbf{Q3}] How to estimate the stone-mortar interface strength of a simplified surfaces?
        \end{itemize}
    \end{textblock*}
\end{frame}

\usebackgroundtemplate{\includegraphics[width=1.\paperwidth]{break.pdf}}
\begin{frame}[t]{}
\begin{textblock*}{10cm}(3.7cm,2.9cm)
\Huge\textbf{\textcolor{black}{Why we are shifting to 3D micro-mechanical modeling?}}
\end{textblock*}
\end{frame}
\usebackgroundtemplate{}



\begin{frame}[t]{Motivation: \textcolor{myviolet}{{\textbf{Q1}}}}\vspace{10pt}
    \begin{textblock*}{13cm}(1.7cm,2.7cm)
    \begin{enumerate}
        \item Stone masonry walls are usually not homogeneous through the thickness\vspace{10pt}\pause
        \item Leaf-separation effects on the strength capacity\vspace{10pt}\pause
        \item In-plane and out-of-plane behaviours interaction\vspace{10pt}\pause
        \item Internal cracking onsets and 3D crack paths (cannot be captured experimentally)
    \end{enumerate}
    \end{textblock*}
\end{frame}

\subsection{Methodology}
\begin{frame}[t]{The study main phases}\vspace{10pt}
    \begin{textblock*}{13cm}(3.8cm,0.7cm)
        \includegraphics[height = 0.6\textwidth]{loop.pdf}
    \end{textblock*}
\end{frame}


\usebackgroundtemplate{\includegraphics[width=1.\paperwidth]{break.pdf}}
\begin{frame}[t]{}
\begin{textblock*}{13cm}(3.5cm,4cm)
\Huge\textbf{\textcolor{black}{How to arrange stones?}}
\end{textblock*}
\end{frame}
\usebackgroundtemplate{}

\begin{frame}[t]{Objective function}\vspace{1pt}
\begin{columns}
\begin{column}{0.49\textwidth}
\begin{overprint}
\onslide<1->\begin{block}{Packing objective}
    \begin{equation*}
        \text{Minimize}~F(\vec{X_i})_{i}~=~\mid\mid \vec{S}_{i} - \vec{S}_{i-1} \mid\mid
    \end{equation*}
    \begin{equation*}
        Fitness\Big(F(\vec{X_i})\Big)_{i} = F(\vec{X_i})_{i}(1 + \xi_{1} P_A)^{\xi_{2}}
    \end{equation*}
\end{block}
\end{overprint}
\end{column}
\end{columns}
\begin{textblock*}{3.2cm}(12.5cm,1.5cm)
    \tiny{
    \begin{itemize}
        \item $S_{i}, S_{i-1}$: locations of $i$ and $i-1$ stones
        \item $\xi_{1}$: penalty multiplier
        \item $\xi_{2}$: penalty exponent
        \item $P_{A}$: penalties summation
    \end{itemize}
    }
\end{textblock*}
\begin{textblock*}{3cm}(12.5cm,1.55cm)
    \includegraphics[height = 0.6\linewidth]{brace.pdf}
\end{textblock*}

\end{frame}


\subsubsection{Generated sample}

\usebackgroundtemplate{\includegraphics[width=1.\paperwidth]{break.pdf}}
\begin{frame}[t]{}
\begin{textblock*}{10cm}(4.3cm,3.2cm)
\Huge\textbf{\textcolor{black}{The final results}}
\end{textblock*}
\begin{textblock*}{10cm}(3.7cm, 5.2cm)
\normalsize\textbf{\textcolor{black}{Sample size: 700 mm x 400 mm x 500 mm}}
\end{textblock*}
\end{frame}
\usebackgroundtemplate{}


\section{Detailed plan}


\begin{frame}{Interface identification: \textcolor{myviolet}{\textbf{Q3}}}\vspace{4pt}
    \begin{textblock*}{10cm}(0.3cm, 2.5cm)
        \includegraphics[height = 0.5\textwidth]{Interface.pdf}
    \end{textblock*}
    \begin{textblock*}{10cm}(6.7cm, 2.0cm)
    \begin{enumerate}
        \item Prepare real samples of stones and mortar\vspace{10pt}\pause
        \item Scan the interface surface using the laser scanner\vspace{10pt}\pause
        \item Test the samples (direct shear test)\vspace{10pt}\pause
        \item Apply different smoothing cycles on the interfaces\vspace{10pt}\pause
        \item Use the soft computing algorithms to identify the smoothed interfaces inputs (Benchmark)
    \end{enumerate}
    \end{textblock*}
\end{frame}

\begin{frame}{Call another section}
    \input{Sections/SomeSection.tex}
\end{frame}

% -----------------------References
% Thank you slide should be here
\subsection{Thank you!}
\usebackgroundtemplate{\includegraphics[width=1.\paperwidth]{break.pdf}}
\begin{frame}[t]{}
\begin{textblock*}{10cm}(3.2cm,4cm)
\Huge\textbf{\textcolor{black}{Merci de votre attention}}
\end{textblock*}
\end{frame}
\usebackgroundtemplate{}

% -----------------------References
\section{Bibliography}
% \begin{frame}[allowframebreaks]{\\References}\vspace{4pt}
\begin{frame}{References}\vspace{4pt}
\tiny{\printbibliography}
\end{frame}
\normalsize

\end{document}